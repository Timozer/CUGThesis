\chapter{更新日志}



\section{Version 0.1}
\label{sec:version_0_1}

2018年 3月14日 星期三 14时05分07秒 CST

本模板从12日着手准备开发, 到现在已经将论文所要求的页面已经基本实现了, 除了作者简介页面和参考文献.
虽然页面基本实现了, 而且一些格式也设置了, 但仍然没有达到论文规范的要求, 因此在接下来的版本中我
会逐渐将没有实现的实现了, 把格式都调好了.

TODO list:
\begin{itemize}
    \item 题名页, 原创声明, 导师承诺书, 授权书, 摘要这些页面继续调整一下, 让人看起来更舒服一些;
    \item 目录格式;
    \item 关于论文中字体格式的调整;
    \item 模板使用字体的选择;
    \item 小工具的编写;
    \item 还没想好\dots
\end{itemize}

2018年 3月15日 星期四 11时18分01秒 CST

修正字号, 因为正文默认使用小四字体, 即在 \CTeX{} 中设置为 12pt, 所以关于 Huge, huge, LARGE 等这些命令
所对应的字体变大了. 而之前使用的定义有误, 所以重新定义字号大小.

设置目录的格式完成. 关于目录的格式问题: 规范里要求一级节标题和二级节标题左缩进一个汉字和两个汉字的距离,
但是章标题和节标题的字号大小却不一样, 所以排版出的效果不是很好. 因此, 我擅自将一级节标题的缩进改为 1.2
个汉字, 二级节标题改为 2.5 个汉字. 

TODO list:
\begin{itemize}
    \item 模板使用字体的选择;
    \item 小工具的编写;
    \item 说明文档的完善;
    \item 简介页面的制作;
    \item 参考文献页面的制作;
    \item 还没想好\dots
\end{itemize}

OK, 目前这个版本实现的功能就是这么多, 其他功能下个版本再添加. 

\section{Version 0.2}
\label{sec:version_0_2}
初步完成了简历页面的制作, 模板使用字体暂时就这些吧.

该版本已经基本完成, 目前已经是一个可以使用的模板了, 不过说明文档的欠缺, 使得
在使用时有一定的难度. 之后我会慢慢的补充文档, 编程好的同学可以自己查看我的
类文档, 来获得使用方法.

Developed!

\section{Versino 0.3}
\label{sec:versino_0_3}

2018-12-18 18:27

增加了第二导师命令;前面页面标点换成全角。

2018年 3月16日 星期五 18时44分24秒 CST

这一版本中主要是完善说明文档, 基本将第二章写完了.

在更新说明文档的过程中, 又对模板类进行了一次结构优化, 使用在使用的过程中需要输入
的代码量更少, 这样大大简化了论文写作过程, 是作者专注于论文的内容, 而不是各种各样
内容标记代码.

此外, 这一版在写说明文档的时候还附带了写了一个简化代码的小工具. 比如说插图和表格,
在论文写作规范中要求标题的位置不同, 你自己使用浮动体环境的时候需要自己有意设置标题位置,
对此我做了一个封装, 使得你不用关心他们的标题到底应该放在哪儿.

还在交叉引用的标号左右增加了一点距离, 使得看起来更美观.

2018年 3月17日 星期六 14时36分07秒 CST

修正一个页眉的错误, 奇数页页眉要求是要显示博士学位论文, 硕士学位论文这几个字的. 我
之前看的是附件, 没有主要到这一要求.

修正一个BUG\@. 博士选项开启时候会出现一个问题, 已经解决了.

2018年 3月17日 星期六 22时53分42秒 CST

痛定思痛后终于下定决心换字体了, \CTeX{} 默认设置的字体也太难看了. 我将宋体和黑体换成了
思源宋体和思源黑体.

这两款字体免费的, 在使用的时候不需要考虑版权的问题. 下载地址见脚注\footnote{\url{https://github.com/adobe-fonts}}.

2018年 3月19日 星期一 09时13分52秒 CST

代码测试: \tilcode{\$~\dots~\$}

\section{Versino 0.4}
\label{sec:versino_0_4}

2019-12-12 22:22

添加在线编译器 overleaf 的使用方法.

\section{Versino 0.5}
\label{sec:versino_0_5}

2022-02-14 22:26

1. 奇数页页眉去掉(武汉)

2. 增加figtoc, tabtoc 用于控制图清单和表清单,默认都为 false, 即不输出图表清单目录,
   如果设置 figtoc,则输出图清单目录,如果设置tabtoc,则输出表清单目录

3. 页眉从第一章开始

4. 图表清单页页眉只提供页码


\endinput